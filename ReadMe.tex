%ÄR BASERAT PÅ DENNA MALL:
% https://www.overleaf.com/latex/templates/uioposter/tsmrpnztthrr
%%%%%%%%%%%%%%%%%%%%%%%%%%%%%%%%%%%%%%%%


%Hur gör man block? Se exempel nedan.
%%%%%%%%%%%%%%%%%%%%
%                   Svart
%    \begin{blackblock}{Does it come in black?}
%        Sure, use an \textbf{exampleblock}!
%    \end{exampleblock}

%                   Röd
%    \begin{redblock}{How do you make it red?}
%        Use an \alert{alertblock}!
%    \end{alertblock}

%              VANLIGT BLOCK
%    \begin{block}{Method}
%        \lipsum[1]
%    \end{block}
%%%%%%%%%%%%%%%%%%%

%Om du vill justera blocken, använd desssa i documentklassen:

%Om man vill justera blocken.
%\setlength{\marginparwidth}{100cm}
%\setlength{\textwidth}{0.9\textwidth}
%\setlength{\oddsidemargin}{10cm}
%\setlength{\topmargin}{-2.5pc}
%\setlength{\headsep}{10cm}
%\setlength{\parskip}{10cm plus 100cm}

%%%%%%%%%%%%%%%%%%%%%%%%%%%%%%%%%%%%%%%%%%%

%om du vill bädda in filer gör som nedan
% \attachfile[icon=Paperclip]{Embeds/TonyMerSpec.pdf}


%%% GRabba/catcha en filklump
%\CatchFileBetweenTags\temptoken{Summary}{cutecode} %
%\setbox0=\vbox{\the\temptoken }%  skip part1

% - eller
%\ExecuteMetaData[Summary]{cutecode}


%% Remove footline:
%\setbeamertemplate{footline}{}
%\setbeamertemplate{}{}