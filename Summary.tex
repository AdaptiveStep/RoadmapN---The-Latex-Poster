
%\documentclass[portrait.tex]{subfiles}

\begin{document}
%<*title>
Roadmap \arabic{NumFiles}
%</title>

%<*subtitle>
Gör klart program och rapport.
%</subtitle>

%<*subsubtitle>
(1) Redovisning Nästa vecka på: \textbf{Allt}. (1) Prov Nästa vecka på: \textbf{Allt}.
%</subsubtitle>

%Ligger under kodrutan
%<*summary>
ee
%</summary>

%<*cutecode>
%hello from cutecode  - FUNKAR EJ
%</cutecode>



%<*boxes>
    \begin{blackblock}{Onsdagsmöte - Sista}
        SISTA PROJEKT MÖTET \\ För denna kurs - ONSDAG \\ imorgon  13 nov. \hfill kl 10.00-12.00 som vanligt.
    \end{blackblock} 

    \begin{blackblock}{Status: \hfill  Menyer/Navigering - "GUI"}
        \begin{itemize}
            \item KOD - Kundmenyer\hfill        Andre,Hariz \done
            \item KOD - Bagarmenyer\hfill       Jakob,Elchin \waiting
            \item KOD - Kassörskamenyer\hfill   Jonathan,Marcus \waiting
            \item KOD - InfoMeny\hfill          Jonathan,Marcus \done
        \end{itemize}
    \end{blackblock}    
    
    \begin{blackblock}{Status: \hfill  Implementering}    
        Kod Status för de fyra terminalernas funktionalitet.
        \begin{itemize}
            \item KOD - KundTerminal\hfill Andre,Hariz      \waiting
            \item KOD - BagarTerminal\hfill Jakob,Elchin    \waiting
            \item KOD - KassörskaTerminal\hfill Jonathan,Marcus \waiting
            \item KOD - InfoMeny\hfill Jonathan,Marcus          \waiting
        \end{itemize}
    \end{blackblock}    
    
    \begin{redblock}{Terminalerna är färdiga \hfill 14 NOV}
        På \textbf{TORSDAG} förväntas samtliga program vara färdiga ASAP. Bara rapporten skall återstå efter torsdagen.
    \end{redblock} 
    
    \begin{redblock}{Presentation \hfill 19 NOV}
        7 dagar kvar. Presentation på tisdag. Lär er programmet, allt ska kunna redovisas, varenda rad. 
    \end{redblock} 
    
        \begin{redblock}{PROV \hfill 21 NOV}
Nästa vecka är provet. Följande ingår inte enligt möte med ayssen: Events, Async, Svåra Delegat. \\ Följande ingår inte i prov enligt Jesse: Unsafe code, UML. \\Man får endast ha Bok+Internet som hjälpmedel till kod-delen. Inga hjälpmedel till teoridelen.
    \end{redblock} 

    \begin{greyblock}{Github}
             \begin{itemize}
                \item https://github.com/AdaptiveStep/PizzaPalatsetG1.git 
             \end{itemize}
    \end{greyblock}

   \begin{greyblock}{Status: \hfill  RAPPORT}    
       Nya uppgifter. Alla i gruppen skall göra egna rapporter. Rapporterna skall lämnas in separat med egna åsikter. \\ \\
       \textbf{Förslag på Delmål för rapport:}
        \begin{enumerate}
            \item Förarbete\&Diskussion 
            \item Samla ihop alla bilder
            \item Bestämma rubriker
            \item Introduktion
            \item Fylla ut rubriker
            \item Sammanfatta allt
            \item Gör allt snyggt
        \end{enumerate}
    \end{greyblock}  

    \begin{greyblock}{Jesses Förslag till Rubriker}
    Under lektionen fick vi följande förslag till rapport.\\ \\
    \textbf{Förslag på Rupriker för rapport:}

    \begin{enumerate}
        \item Beskriv varje milstolpe i utvecklingen som separat kapitel
        
        \item Beskriv vilka tekniska problem ni har haft och hur ni löste dom
        
        \item Beskriv vilka gruppdynamiska problem ni har haft och hur ni löste dom.
        
        \item Reflektion om vad föreläsningarna saknade som ni skulle haft nytta av i grupparbete
        
        \item Innan ni började, vad trodde ni skulle bli problematiskt och var det problematiskt?
        
        \item Vad hade ni inte förväntat er vara problematiskt som var problematiskt
        
        \item Hur ser ni på C\# språket efter projektet?
        
        \item Hur ser ni på visual studio efter projektet?
        
        \item Hur ser ni å GIT efter projektet, har ni haft nytta av GIT?
        
        \item Var UML/Balsamiq diagrammen användbara och tycker ni att ni använde den på bra sätt?
        
        \item Hur känns rollen som systemutvecklare efter detta första projekt?
        
        \item Kädes uppgifter för svår, för lätt eller lagom med hänsyn till både kursens innehåll och arbetsbörda.
        \end{enumerate}
    \end{greyblock}




 
%</boxes>

%<*names>
    .Jacob.Elchin.André.\\
    Marcus.Jonathan.Hariz\\ 
    - Skapat: \today
%</names>

\end{document}